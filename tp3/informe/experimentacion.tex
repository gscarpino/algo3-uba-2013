\section{Experimentación General}


\subsection{Comparando con el Exacto}

\quad Vamos a comparar todas nuestras implementaciones y a calcular las distancias con respecto al algoritmo exacto. Debido al costo del exacto se realizaron tests de 5 nodos a 16 con 50 repeticiones para cada cantidad. Definimos distancia como la diferencia entre el resultado exacto y el de la heurística correspondiente. Esto nos da una idea de \textit{que tan mala es la heurística}.


\quad Comparamos primero con grafos generados aleatoriamente, luego con grafos estrellas no uniformes y grafos \textit{web}.


\begin{figure}[H]
	\centering
	\includegraphics[scale=0.6]{distancia-Azar.png}
\end{figure}

\begin{figure}[H]
	\centering
	\includegraphics[scale=0.6]{distancia-Star.png}
\end{figure}

\begin{figure}[H]
	\centering
	\includegraphics[scale=0.6]{distancia-Web.png}
\end{figure}

\quad Se ve claro como a medida de que se incrementan los nodos en los grafos la calidad de las soluciones obtenidas empeoran.

\quad Con respecto a la heuristica golosa es la peor de las heuristicas, sin embargo como se vera más adelante es la de menor costo temporal. El empeoramiento (distancia) es lineal a la solución exacta en las 3 familias de grafos.

\quad Con la búsqueda local pasa algo similar salvo que por los datos obtenidos podriamos decir que es prácticamente la mitad de peor que el goloso. 

\quad En cambio, con la metaheurística GRASP se obtiene distancias casi nulas con respecto a la solución exacta. Siendo en promedio menor a una para este rango de nodos de grafos que experimentamos (en las 3 familias).

\quad

\quad Ahora vamos a mostrar los datos anteriores pero comparando cada heuristica consigo misma fijándonos la distancia con la solución exacta.

\begin{figure}[H]
	\centering
	\includegraphics[scale=0.6]{distancia-Goloso.png}
\end{figure}

\quad Prácticamente el goloso se comporta igual con los tres tipos de grafos, es decir, que la calidad de las soluciones obtenidas no depende de estas familias de grafos. Podemos determinar que a pesar de la similitud, esta heurística es levemente peor con los grafos Web de 4 vértices.

\quad

\begin{figure}[H]
	\centering
	\includegraphics[scale=0.6]{distancia-BLocal.png}
\end{figure}

\quad Vemos un comportamiento menos estable que con el goloso con respecto a las distintas familias de grafos. Sin embargo la tendencia es clara, lineal. No se puede determinar una familia para el cual esta heurística es peor.

\begin{figure}[H]
	\centering
	\includegraphics[scale=0.6]{distancia-GRASP.png}
\end{figure}

\quad Esta metaheurística es la menos estable de las 3. Varia mucho la distancia, presentando varios picos. Creemos que se debe a los criterios de parada definidos aunque hay que tener en cuenta que el promedio máximo es 0.5 menor a 1. Se podria decir que esta metaheuristica se comporta peor con las familias de grafo Web de 4 vertices porque con los grafos \textit{más grandes} es la que produce mayor distancia de la solución exacta.

\quad



\subsection{Comparando con Grasp}

\quad Dejando de lado el algoritmo exacto se puede experimentar con una mayor cantidad de nodos.

\quad Comparamos la heuristica golosa, la búsqueda local y GRASP.

\quad 

\subsubsection{Distancia}

\begin{figure}[H]
	\centering
	\includegraphics[scale=0.6]{distancia-Grasp-Azar.png}
\end{figure}

\begin{figure}[H]
	\centering
	\includegraphics[scale=0.6]{distancia-Grasp-Star.png}
\end{figure}

\begin{figure}[H]
	\centering
	\includegraphics[scale=0.6]{distancia-Grasp-Web.png}
\end{figure}

\subsubsection{Costo}

\quad Algo...

\begin{figure}[H]
	\centering
	\includegraphics[scale=0.6]{timingAzar.png}
\end{figure}

\begin{figure}[H]
	\centering
	\includegraphics[scale=0.6]{timingStar.png}
\end{figure}

\begin{figure}[H]
	\centering
	\includegraphics[scale=0.6]{timingWeb.png}
\end{figure}

\quad

\quad En este gráfico podemos observar como GRASP a pesar de obtener resultados mucho mejores que las otras heuristicas, consume mucho más tiempo de procesamiento.

\quad

\quad Si bien la búsqueda local cuesta apenas más que la golosa, en comparación con GRASP, se obtiene muchos mejores resultados.
