
\documentclass[spanish, 12pt,a4paper]{article}
%\usepackage[spanish]{babel}
\input{macrostp2.tex}
\usepackage{hyperref}
\usepackage{color}
\usepackage{graphicx}
\usepackage{graphics}
\usepackage{clrscode3e}
\usepackage{amsthm}
\usepackage{caption}
\usepackage{subcaption}
\usepackage{caratula}
\usepackage{fancyhdr,lastpage}
\usepackage[paper=a4paper, left=1.4cm, right=1.4cm, bottom=1.4cm, top=1.4cm]{geometry}
\usepackage[table]{xcolor} % color en las matrices
\usepackage[font=small,labelfont=bf]{caption} % caption de las figuras en letra mas chica que el texto
				    
\color{black}

%%%PAGE LAYOUT%%%
\topmargin = -0.5cm
\voffset = 0cm
\hoffset = 0em
\textwidth = 37em
\textheight = 130 ex
\oddsidemargin = 4 em
\parindent = 2 em
\parskip = 3 pt
\footskip = 7ex
\headheight = 20pt
\pagestyle{fancy}
\lhead{AyED	 III - 2013 C1 - Trabajo Pr\'actico 3} % cambia la parte izquierda del encabezado
\renewcommand{\sectionmark}[1]{\markboth{#1}{}} % cambia la parte derecha del encabezado
\rfoot{\thepage}
\cfoot{}
\numberwithin{equation}{section} %sets equation numbers <chapter>.<section>.<subsection>.<index>

%Lo siguiente controla el ancho de las figuras (principalmente para el texto de los captions)
%\newcommand{\figurewidth}{.9\textwidth}
\newcommand{\figurewidth}{1\textwidth}

\newcommand{\tuple}[1]{\ensuremath{\left \langle #1 \right \rangle }}
\newcommand{\Ode}[1]{\small{$\mathcal{O}(#1)$}}
\newtheorem{teorema}{Teorema}[section]


%El siguiente paquete permite escribir la caratula facilmente
\hypersetup{
  pdftitle={ Algoritmos III - TP1 },
  colorlinks,
  citecolor=black,
  filecolor=black,
  linkcolor=black,
  urlcolor=black 
}

\materia{Algoritmos y Estructuras de Datos III}

\titulo{Trabajo Práctico 2}

\subtitulo{Informe y análisis de resultados.}

%\grupo{Grupo 8}
 
\integrante{Benitti, Raul}{592/08}{raulbenitti@gmail.com}
\integrante{Mengarda,Lucas }{787/10}{l.j.mengarda@gmail.com}
\integrante{Scarpino, Gino}{392/08}{gino.scarpino@gmail.com}
\integrante{Vallejo, Nicolás}{500/10}{nico\_pr08@hotmail.com}
 
\begin{document}
{ \oddsidemargin = 1em
	\headheight = -20pt
	\maketitle
}
	\tableofcontents
	\newpage
	%descripcion de situaciones reales
	\section{Descripción de situaciones reales}
En este trabajo se presenta otra variante al problema de coloreo, conocida como Coloreo de M\'aximo Impacto (CMI).
Se define el \textbf{impacto} de un coloreo C sobre un grafo G = (V, E) como el n\'umero de aristas $vw$ $\in $ E tal que el color de $v$ es igual al de $w$.
Dados dos grafos G y H definidos sobre el mismo conjunto de v\'ertices, CMI consiste en encontrar un coloreo C de G que al ser aplicado en H maximice el impacto de C en H. 
En general, observamos que los problemas que se pueden modelar tiene las siguientes caracter\'isticas:
\begin{itemize}
	\item se tiene un mismo conjunto de elementos, con dos criterios para relacionarlos entre si;
	\item se quiere asignar (o particionar) los elementos de forma tal que se maximiza la cantidad de elementos relacionados seg\'un criterio, mientras que se respeta el otro.
\end{itemize}
	
Presentamos aqu\'i algunos ejemplos de esto aplicado a situaciones que pueden darse en la vida real.


\subsection{Ejemplo 1}
En una facultad, se tiene \textsl{n} materias que corresponden a alguna de \textsl{m} \'areas de estudio.
Las materias pueden tener horarios solapados, de manera que no siempre es posible utilizar el mismo aula para dar dos materias distintas el mismo d\'ia.
Adem\'as, cada materia necesita que en su aula se encuentren elementos de trabajo espec\'ificos, seg\'un el \'area a la que corresponda. 
A fin de poder equipar las aulas de la mejor manera reduciendo la cantidad de equipamiento necesario, se desea asignarlas a las materias de forma tal que se maximice la cantidad de materias del mismo area que se dicten en un mismo aula.

Podemos resolver el problema planteando dos grafos G y H con las materias como nodos, tales que:
\begin{itemize}
	\item en G, los nodos est\'an relacionados por una arista si las materias tiene horarios solapados;
	\item en H, los nodos est\'an relacionados si pertenecen al mismo \'area de estudio,  
	\item los colores asignados representan las aulas asignadas a las materias.
\end{itemize}
En este caso, un coloreo de m\'aximo impacto ser\'a aquel que en G no asigne las mismas aulas a materias con horarios solapados, mientras que en H, maximiza la cantidad de materias del mismo \'area que comparten el mismo aula.

\subsection{Ejemplo 2}
Un centro de salud tiene pacientes internados en distintas sedes.
Los enfermeros del centro deben atender a los pacientes en horarios espec\'ificos, por lo si dos pacientes coinciden en estos horarios, el mismo enfermero no puede tratarlos. Debido a esto, puede ser necesario que los enfermeros se muevan entre las distintas sedes.
A fin de minimizar este movimiento de personal, se desea asignar los enfermeros a los pacientes tal que se maximice la cantidad de pacientes que cada uno debe tratar en una misma sede.

Aqu\'i, en el modelado consideramos a los pacientes como nodos de los grafos G y H, tal que 
\begin{itemize}
	\item en G, los nodos est\'an relacionados si los pacientes necesitan ser atendidos en el mismo momento;
	\item en H, los nodos est\'an relacionados si los pacientes se encuentran en la misma sede.  
	\item los colores asignados representan a los distintos enfermeros
\end{itemize}
Entonces, un CMI representa la asignaci\'on de enfermeros a los pacientes de manera que ningun enfermero sea asignado a dos pacientes si estos requieren atenci\'on en el mismo momento, mientras que se maximiza la cantidad de pacientes que cada enfermero ve en cada sede. 

	\newpage
	%algoritmo exacto
	\section{Algoritmo Exacto}


\subsection{Algoritmo}
Antes de desarrollar la explicaci\'on, definimos algunos elementos que utilizaremos luego.
Sean G = (V, E$_G$), H = (V, E$_H$) dos grafos con V el mismo conjunto de nodos, y sea $n$  = $|$V$|$. 
Consideramos que los nodos de V se encuentran numerados de alguna manera con los n\'umeros de 1 a n.
Sean los n colores representados por el conjunto C = \{1, 2, ..., n\}; un coloreo ser\'a representado por una lista $f$ de n valores, cada uno de ellos pertenenciente al conjunto C, donde el nodo $i$ es pintado con el color en la posici\'on \textsl{f}$_i$.
Buscamos desarrollar un algortimo exacto que nos permita encontrar alg\'un coloreo \textsl{f} de G que use los colores de C, tal que el impacto (tal como se define en el enunciado) de \textsl{f} en H sea m\'aximo. 

La idea general del algoritmo es simple: generar todos los coloreos del grafo G, y por cada uno que sea legal, calcular el impacto en H y guardar aquel con el que el valor obtenido sea m\'aximo. 

Para genererar todos los coloreos posibles, vamos a partir de la siguiente funci\'on recursiva:

\begin{algorithm}[H]
\caption{} 
\begin{codebox}
\Procname{$\proc{coloreos}(nodo, coloreo)$}
\li \If \textit{nodo} $\leq$ \textsc{length}(\textit{coloreo}) \Do
\li 		\textsl{hacer algo con el coloreo}		
		\End
\li	\Else \Do
\li			\For color \textbf{in} Colores \Do	
\li					\textit{coloreo}[\textit{nodo}] = color
\li					\textsc{coloreos}(nodo+1, coloreo)	
				\End
		\End
\End
\end{codebox}
\end{algorithm}

Esta funci\'on genera todos las combinaciones posibles de los colores de C (\textsl{Colores}) en los nodos de G.
%: en cada posici\'on del coloreo (es decir, para cada nodo del grafo) pone cada uno de los colores de \textsl{Colores}, y para cada uno de ellos, completa las siguientes posiciones del coloreo.
Sin embargo, \'esta es una tarea muy costosa, pues la cantidad de resultados posibles es de orden exponencial. Por ejemplo, si el grafo tiene n nodos, obtendr\'iamos al menos $n!$ coloreos: podr\'iamos colorear cada nodo de un color diferente, y despu\'es permutar los colores entre todos lo nodos. 
M\'as all\'a de esto, dadas las caracter\'isticas del problema planteado, veremos algunos recursos que nos permitiran reducir dr\'asticamente la cantidad de casos a analizar.


En primer lugar observamos que, para un grafo G dado, solo nos interesa analizar los coloreos que son legales (es decir, aquellos en los que si dos nodos son adyacentes, entonces est\'an pintados de colores distintos). Esto nos da una pauta sobre como aplicar podas a medida que vamos generando los coloreos: al pintar un nodo i (es decir, elegir un color para el posici\'on i del coloreo), utilizaremos solo los colores que produzcan un coloreo legal hasta ese momento.

Por ejemplo, en la figura \ref{fig:ejemploSoloColoreosNoValidos} mostramos dos pasos en la generaci\'on de un coloreo para un grafo de 5 nodos. Podemos ver que si tenemos el nodo 2 pintado de color negro, al pintar el nodo 3 tambi\'en de color negro obtendremos un coloreo ilegal para cualquier otra combinaci\'on de colores en los nodos siguientes. 

\begin{figure}[H]
	\centering
	\includegraphics[scale=1]{ejemplo-coloreos-no-validos.png}
\label{fig:ejemploSoloColoreosNoValidos}
\caption{Ejemplo de coloreo de un nodo que lleva a coloreos ilegales.}
\end{figure}

Por lo tanto, podemos aplicar una una poda considerando solamente los colores que sean legales con respecto a los de los nodos pintados con anterioridad:

\begin{algorithm}[H]
\caption{} 
\begin{codebox}
\Procname{$\proc{coloreos}(nodo, coloreo)$}
\li \If \textit{nodo} $\leq$ \textsc{length}(\textit{coloreo}) \Do
\li 		\textsl{hacer algo con el coloreo}		
		\End
\li	\Else \Do
\li			\For color \textbf{in} Colores \Do
\li		 		\If color legal para el nodo \textsl{nodo}\Do	
\li					\textit{coloreo}[\textit{nodo}] = color
\li					\textsc{coloreos}(nodo+1, coloreo)	
					\End
				\End
		\End
\End
\end{codebox}
\end{algorithm}
 
Por otro lado, notamos que existen coloreos que son equivalentes (en el sentido en que uno puede obtenerse a partir del otro por medio de un renombramiento de sus colores), y por lo tanto, una vez analizado un caso, los dem\'as solo aportan informaci\'on redundante. 
Por ejemplo, supongamos que tenemos un grafo de 5 nodos 3-coloreable, y sean C = \{\textsl{gris} = 1, \textsl{rayado} = 2, \textsl{negro} = 3\} los colores. 
Dos coloreos v\'alidos son 1-2-3-1-2 y 2-1-3-2-1, como se muestra en la figura \textbf{\ref{fig:ejemploRepeticionColoreo}}. 
Sin embargo, f\'acilmente se ve que al intercambiar las etiquetas de los colores \textsl{1} y \textsl{2}, el primer coloreo puede transformase en el segundo, y el segundo puede transformase en el primero. 

\begin{figure}[H]
\label{fig:ejemploRepeticionColoreo}
\caption{Ejemplo de dos coloreos legales que resultan equivalentes.}
	\centering
	\includegraphics[scale=1]{ejemplo-mismos-coloreos.png}
\end{figure}


%Dicho m\'as formalmente, existen pares de coloreos f y f' para los podemos establecer una biyecci\'on $\Phi : \; C \rightarrow C$ tal que  $f'(v) = \Phi(f(v))$ y $f(v) = \Phi(f'(v))$.

Recordemos que numeramos los nodos y los colores con los n\'umeros de 1 a n, y dejemos de lado, por el momento, la condici\'on de legalidad.
Definamos un conjunto de coloreos F donde cada coloreo f es un vector 
$$f = [f(1), ... , f(n)]$$


tal que, para cada uno, $f(v)$ cumple

\begin{itemize}
	\item $m$ es el m\'aximo en $f[1...v-1]$ (0, si $v-1 \leq 0$)
	\item $1 \leq f(v) \leq m+1$
\end{itemize}

Es decir, para pintar el nodo v, se usan $m+1$ maneras distintas, usando los colores ya usados, de 1,2...$m$ o un color nuevo, $m+1$.

\begin{figure}[H]
\label{fig:ejemploRepeticionColoreo}
	\centering
	\includegraphics[scale=1]{ejemplo-coloreos-distintos.png}
\caption{Ejemplo de un conjunto de coloreos seg\'un la caracterizaci\'on dada. Cada camino desde la ra\'iz a una hoja es un coloreo. Por ejemplo, 1-1-1-1, 1-2-1-2 y 1-2-3-4.}
\end{figure}


Por la forma en que estan construidos, resulta que dos coloreos f y f' cualesquiera de F no son equivalentes. 

Adem\'as, cualquier otro coloreo que no est\'e en F es equivalente a uno que si lo est\'a.

%Para evitar analizar estos casos de m\'as, definimos la siguiente manera para determinar los colores a usar en cada momento:
Entonces, teniendo esto en cuenta, para evitar analizar los casos de m\'as definimos una forma m\'as restrictiva para armar los coloreos.
El pseudocódigo del algoritmo es el siguiente:

\begin{algorithm}[H]
\caption{} 
\begin{codebox}
\Procname{$\proc{colorear}(nodo, G, H, coloreo, solucion)$}

\li \If $nodo \; \geq \; G.cantNodos$ \Comment{se pintaron todos los nodos de G} \Do
\li	$impacto$ $\gets$ impacto del coloreo en H
\li		\If $impacto\; > \; solucion.impacto$ \Do
\li			$solucion.impacto \gets impacto$
\li			$solucion.coloreo = coloreo$
			\End
	\End
\li \Else \Do
\li		$maxColor$ $\gets$ maximo color usado hasta el momento 	
\li		\For c \textbf{from} 1 \textbf{to} maxColor \Do
\li			\If es legal pintar $nodo$ de color c \Do
\li					$nuevoColoreo \; = \; coloreo$
\li         $nuevoColoreo[nodo] \gets c$
\li         $colorear(G, H, nuevoColoreo, solucion)$
             \End
         \End
\End
\end{codebox}
\end{algorithm}


La funci\'on que resolver\'a el problema planteado es

\begin{algorithm}[H]
\caption{} 
\begin{codebox}
\Procname{$\proc{maximoImpactoExacto}(Grafo$ G$, Grafo$ H$)$}
\li Sea $solucion$ un vector de n+1 elementos
\li Sea $coloreo$ un vector de n elementos
\li $solucion[0] \gets 0$
\li $coloreo[0] \gets 1$
\li $colorear(1,g,h,coloreo,solucion)$
\li	return $solucion$
\End
\end{codebox}
\end{algorithm}



\subsection{Análisis de complejidad}

\indent Al momento de hacer este análisis de complejidad se tuvieron en cuenta algunas consideraciones.\\
\indent En primer lugar, llamaremos $m$ al máximo entre la cantidad de aristas del grafo G y del grafo H  y $n$ a la cantidad de nodos de dichos grafos.\\
\indent En segundo lugar, en el análisis de complejidad de la función $colorear$ se definirá $k$ como la cantidad de nodos que quedan por pintar hasta ese paso de la recursión, en contraste con la implementación donde la recursión es , por así decirlo $hacia arriba$, significando esto que se inicia desde el primer nodo y se va hacia el último.\\

\indent Analicemos primero la función $colorear$. Como mencionamos anteriormente, consideraremos la recursión en la cantidad de nodos que quedan por colorear.

\indent El caso base será cuando no hayan más nodos por pintar. En dicho caso, el algoritmo simplemente calcula el impacto de dicho coloreo en el grafo H y en caso de ser el de máximo impacto hasta el momento se reemplaza la solución anterior por la nueva. Esto cuesta O(n+m), que es lo que cuesta calcular el impacto en H.\\
\indent Para el caso en el que la cantidad de nodos a pintar sea distinta de cero, el algoritmo calcula los posibles colores con los que pintar el nodo. Esto lo hace buscando cuál es el máximo color usado hasta el momento. El nuevo nodo podrá ser pintado de los colores usados anteriormente o del máximo color usado hasta el momento + 1, es decir pintándolo de un nuevo color. Esto se calcula en tiempo O(n).Luego se llamará a la función recursivamente una cantidad de veces igual al máximo color a utilizar. Ese valor se puede acotar para todos los casos por n-k+1.\\
\indent Además se chequea que agregar ese color genere un coloreo válido, y eso cuesta O(cantidad de vecinos del nodo), que lo podemos acotar por la cantidad de aristas de G, es decir O(m). Luego se hace la llamada recursiva para la instancia inmeditamente menor.\\
\indent Pasando en limpio, en el paso $k$ el algoritmo cuesta (n-k+1)*(n+m + T(k-1)).\\
\indent Es decir:\\

\indent T(0)=n+m\\
\indent T(k)=(n-k+1)*[(n+m)+ T(k-1)]\\

donde $k$ es la cantidad de nodos que quedan por pintar.\\


\indent Veamos entonces cuánto cuesta pintar todos los nodos.\\
\indent Basado en la definición que dimos antes, si tenemos que pintar todos los nodos, estamos en el caso T(n).\\

\indent T(n)=(n-n+1)*(n+m + T(n-1))\\

\indent Desarrollemos T(n):\\


\begin{centering}
T(n)=(n-n+1)*(n+m + T(n-1))=\\
= n+m + [2*(n+m) + 2*T(n-2)]=\\
=(n+m)+2*(n+m) + 2*[3*(n+m)+ 3* T(n-3)]=\\
=(n+m)+2*(n+m)+ 2*3*(n+m) + 2*3 T(n-3)=\\
=(n+m)+2*(n+m)+ 2*3*(n+m) + 2*3*[4*(n+m)+4*T(n-4)]=\\
=(n+m)+2*(n+m)+2*3*(n+m)+2*3*4*(n+m)+ 2*3*4*T(n-4)=\\
=...=\\
=[ $\sum_{i=1}^{n} i! * (n+m) $] + n! * T(0)= \\
=[ $\sum_{i=1}^{n} i! * (n+m) $] + n! * (n+m)\\
\end{centering}


\indent Es decir que T(n)= [ $\sum_{i=1}^{n} i! * (n+m) $] + n! * (n+m)\\

\indent Conjeturamos entonces que T(n) es O($\sum_{i=1}^{n} i! * (n+m) $).\\

\indent Veámoslo por inducción en la cantidad de nodos por pintar:\\

\indent Queremos ver que existe un $d$ real positivo y $n_{0}$ natural positivo tales que para todo $n\geq n_{0}$ vale que $T(n) \leq d * [\sum_{i=1}^{n} i! * (n+m)] $.

\indent Caso base: n=1 \\

\begin{center}
T(1)= $\sum_{i=1}^{1} i! * (1+m)  1! * (1+m) $ = 2*1!*(1+m) =\\
=2*(1+m) $\leq$ 2*  $\sum_{i=1}^{1} i! * (1+m)$ \\
\end{center}

\indent Es decir que con un $d=2$ nos alcanza.\\


\indent Paso inductivo: Suponiendo que vale que $T(n-1) \leq d * [\sum_{i=1}^{n-1} i! * (n-1+m)] $ quiero ver que vale $T(n) \leq d * [\sum_{i=1}^{n} i! * (n+m)] $ \\


\begin{center}
T(n)=(n-n+1)*(n+m + T(n-1))=\\

= (n+m) + T(n-1)$\leq$\\

$\leq$ por hipótesis inductiva $\leq$\\

$\leq n+m + d * [\sum_{i=1}^{n-1} i! * (n-1+m)] \leq $\\

$\leq n+m + d * [\sum_{i=1}^{n-1} i! * (n+m)] \leq$ \\

$\leq (n+m) *[ (d* \sum_{i=1}^{n-1} i!) + 1] \leq$ \\

$\leq (n+m) *[ (d* \sum_{i=1}^{n-1} i!) + n!] \leq$ \\

$\leq (n+m) *[ (d* \sum_{i=1}^{n} i!)] \leq$\\

$\leq  d* \sum_{i=1}^{n} i!*(n+m)$\\

\end{center}

\indent que es lo que queríamos ver.\\

\indent Luego, colorear cuesta O($\sum_{i=1}^{n} i! * (n+m) $).\\

\indent maximoImpactoExacto cuesta entonces O(n+1 + n + 1 + $\sum_{i=1}^{n} i! * (n+m) $) que es O($\sum_{i=1}^{n} i! * (n+m) $).\\

\subsection{Experimentación y Resultados}

\quad Para realizar testeos y generar resultados lo más diversos posibles definimos 3 familias de grafos:

\begin{itemize}
\item \textbf{Grafos aleatorios:} \quad Primero creamos un grafo de la cantidad de vértices deseada pero sin aristas. Luego recorremos todos los nodos por orden de etiqueta. Por cada nodo, determinamos si esta conectado a cada uno de los otros nodos. Usamos la función \textit{rand()} de la Standar Library de $C++$ con una probabilidad del $ 30\div $ si de que esté unido a un nodo. 

 \quad Vimos que la probabilidad si es mayor se generan casi siempre grafos conexos. Como buscamos que sean lo más diversos posibles, nos pareció un valor razonable donde se generan diversas componentes conexas y hasta incluso grafos conexos.

\quad

\item \textbf{Grafos Estrella no uniformes (Star):} \quad Se parte un nodo central luego se le añaden 4 nodos a los cuales los consideremos externos. Se van a ir agregando nodos hasta la cantidad deseada. Un nodo que se agrega puede estar conectado con un nodo \textit{externo} o el nodo central. Si se agrega al nodo externo se considera como un nuevo nodo externo. Si se conecta con un nodo externo, éste deja de serlo y el nuevo nodo pasa a ser externo. 

\quad Así, todos los nodos menos el central tienen grado uno o dos. Si es externo grado 1, caso contrario 2. El grado del nodo interno varia con cada creación de grafo. La elección a qué nodo se uno se hace aleatoriamente usando la misma función mencionada para los grafos aleatorios.

\quad

\item \textbf{Grafos Web de 4 vértices (Red)\footnote{http://en.wikipedia.org/wiki/List\_of\_graphs}:} \quad estos grafos son un caso particular de grafos $ Web_{s t} $ que son t ciclos de s vértices cada uno conectados entre los ciclos un nodo con un solo nodo del ciclo aledaño. 
\end{itemize}

\quad

\quad \textbf{Aclaración Importante:} Vamos a trabajar con estas familias de grafos con el resto de las experimentaciones de este trabajo práctico. Los resultados (datos procesados y no procesados) de las experimentaciones se encuentran en la carpeta \textit{resultados}.

\quad 

\quad Se midieron los tiempos en corridas de  5 a 15 nodos con 50 repeticiones para cada cantidad de nodos.

\begin{figure}[H]
	\centering
	\includegraphics[scale=0.8]{timingExacto.png}
\caption{Costos}
\end{figure}

\quad Debido al gran costo computacional de este método no se puede experimentar con una mayor cantidad de nodos sin que tome una cantidad de tiempo considerable. Elegir esa cantidad nos permite realizar varias repeticiones para obtener resultados más fiables.

\quad Podemos observar que en este método no influye la diferencia entre las familias de grafos, tomando para cada uno el mismo costo temporal. 

\quad A partir del nodo 14 se ve claramente como la cota teórica sobrepasa a los resultados obtenidos. Éstos muestran una tendencia con una curvatura mucho menos pronunciada que la teórica.

\quad Las mediciones estan en milisegundos, por lo que son despreciables los resultados con pocos nodos que toman pocos milisegundos pero se ve que a medida que aumenta la cantidad de nodos aumenta considerablemente el costo temporal.

	\newpage
	%goloso
	\section{Heurística Constructiva Golosa}

\subsection{Algoritmo}

\begin{algorithm}[H]
\caption{} 
\begin{codebox}
\Procname{$\proc{maximoImpactoGoloso}(Grafo$ g$, Grafo$ h$, double $ porcentaje$)$}
\li
\li	vector$<$unsigned int$>$ res(n+1)
\li res[0] \gets 0
\li
\li vector$<$unsigned int$>$ coloreo(n,1) // todos sus elementos valen 1
\li vector$<$unsigned int$>$ colores(g.gradoMaximo()+1) 
\li
\li \For i desde 0 hasta la cantidad de colores a usar \Do

\li 		colores[i] = i+1
 	\End

\li
\li vector$<$bool$>$ modificados(n,false) //todos sus elementos valen false
\li unsigned int nodo
\li
\li \While $coloreo$ no sea un coloreo legal de g \Do
\li
\li 	nodo \gets $siguienteModificable(G,H,coloreo,modificados,porcentaje)$
\li
\li     \For c desde 1 hasta la cantidad de colores a usar \Do
\li 				  	
\li  			\If pintar a $nodo$ de color c es legal \Do
\li 					coloreo[nodo] \gets $colores[c]$
\li 					salir del for
				\End
\li
		\End
\li 	modificados[nodo] \gets $true$
\li
	\End
\li
\li res[0] \gets $h.impacto(coloreo)$
\li
\li \For i desde 1 hasta n \Do
\li 	res[i+1]=coloreo[i]
\li \End
\li
\li return res
	
\End
\end{codebox}
\end{algorithm}


\begin{algorithm}[H]
\caption{} 
\begin{codebox}
\Procname{$\proc{siguienteModificable}(Grafo$ g$, Grafo$ h$, vector<unsigned$ int$>$ coloreo, $	vector<bool> $ modificados$, double $ porcentaje$)$}
\li
\li 	vector$<$ tupla$<$unsigned int, unsigned int $> >$ posibles
\li     \For i desde 1 hasta n \Do
\li  		\If el nodo i no fue modificado y tiene vecinos \Do
\li				agregar a posibles la tupla $<$impactoNodo(i,h, coloreo), i$>$
			\End
		\End
\li 
\li 	ordenar posibles de menor a mayor
\li 	
\li 	\For i desde 0 hasta el tamaño de posibles -1 \Do
\li
\li 		\For j desde i+1 hasta el tamaño de posibles \Do
\li
\li 			\If posibles[i].first == posibles[j].first \Do
\li 				
\li 					\If grado de posibles[i].second en g$<$ grado de posibles[j].second en g\Do
\li 						swap(posibles[i], posibles[j])
						\End
\li						\Else 
\li 						\If grado de posibles[i].second en g== grado de posibles[j].second en g \Do
\li 							\If grado de posibles[i].second en h $<$ grado de posibles[j].second en h \Do
\li										swap(posibles[i], posible[j])
								\End
							\End
						\End
				\End
			\End
		\End
\li
\li
\li res \gets $elemento elegido al azar de $ 
\li          alguno de los primeros (tamaño de posibles*porcentaje) elementos de posibles
\li
\li return posibles[res].second
\li
\End
\end{codebox}
\end{algorithm}



\begin{algorithm}[H]
\caption{} 
\begin{codebox}
\Procname{$\proc{impactoNodo}($unsigned int $ nodo$$, Grafo$ h$, vector<unsigned$ int$>$ coloreo$)$}
\li
\li unsigned int res \gets 0
\li vector$<$unsigned int$>$ vecinos \gets $vecinos del nodo $nodo$ en h$
\li
\li \For i desde 0 hasta la cantidad de vecinos de $nodo$ en h \Do
\li
\li 	\If coloreo[nodo]==coloreo[vecinos[i]] \Do
\li 		res++
		\End
	\End
\li
\li return res
\End
\end{codebox}
\end{algorithm}



\subsection{Análisis de complejidad}

\indent Comencemos analizando la complejidad de la función impactoNodo.\\
\indent Esta función mira para un nodo el impacto que aporta en H, comparando su color con el de sus vecinos. Dicho nodo tiene en H a lo sumo n-1 vecinos. Luego, impactoNodo cuesta O(n).

\indent Analicemos ahora siguienteModificable. Al inicio comienza iterando sobre la cantidad de nodos de H y si dicho nodo no fue modificado o si no tiene vecinos, se calcula el impacto de cada nodo y se lo agrega a un vector de nodos candidatos a ser modificados.En el peor caso, todos los nodos están sin modificar y tienen vecinos,por lo tanto esto cuesta O($n^{2}$).\\
\indent Luego, se ordena de manera creciente el vector de candidatos de acuerdo al impacto de cada nodo.En el peor caso dicho vector tiene n elementos, pues todos los nodos son modificables y ordenarlos cuesta entonces O(n*log(n)).\\
\indent Luego, se itera sobre la cantidad de elementos de ese vector, esta vez para desempatar los nodos. En el peor caso todos los nodos empatan en el impacto que generan. Desempatarlos a todos cuesta en el peor caso O($n^{2}$) , que el caso en el que se invirtió el orden del vector por desempates.\\
\indent A continuación se elige pseudoaleatoriamente en O(1) uno de los primeros elementos del vector.\\
\indent Pasando en limpio, siguienteModificable cuesta O($n^2$ + n*log(n)+ $n^2$), que es O($n^{2}$).\\

\indent Ahora analicemos maximoImpactoGoloso.\\
\indent Al principio realiza unas cuantas operaciones en O(n). De estas es destacable la creación de un vector de tamaño igual al grado del nodo con grado máximo de G, que refiere a la cantidad de colores a usar. Pero el grado máximo de cada nodo es a lo sumo n-1. Luego crear ese vector cuesta O(n).\\
\indent Luego, se ejecuta un while que a lo sumo itera n veces. Esto es porque en el peor caso tuve que pintar todos los nodos de distinto color hasta obtener un coloreo válido.\\
\inden Dentro de ese while está implícito el chequeo de si el coloreo es válido, que cuesta O(n+m), donde vamos a acotar a m como el máximo entre las aristas de G y de H. Se ejecuta siguienteModificable y se itera luego en la cantidad de colores, costando cada iteración en la cantidad de colores O(n) que es lo que cuesta ver si pintar un nodo de ese color es no coincide con el color de uno de los vecinos de ese nodo, que como mencionamos antes pueden ser n-1.\\
\indent Luego, lo de adentro del while cuesta O(n+m+ $n^{2}$) y el costo total del while es de O(n(n+m +$n^{2}$)), que es O(n*(n+m)+ $n^{3}$).\\
\indent Luego de iterar se calcula el impacto de dicho coloreo en O(n+m).\\
\indent Es decir que en total maximoImpactoGoloso cuesta O(n+n*(n+m)+ $n^{3}$ + n+m).\\
\indent Por lo tanto, maximoImpactoGoloso cuesta O(n*(n+m)+ $n^{3}$).\\
 
\subsection{Experimentación y Resultados}

	\newpage
	%busqueda local
	\section{Heurística de Búsqueda Local}

\subsection{Algoritmo}

%\indent El algoritmo de búsqueda local que implementamos parte de una solución obtenida mediante el algoritmo constructivo goloso mostrado anteriormente. A partir de allí recorre cada nodo y chequea si cambiándole el color por el de alguno de sus vecinos si aumenta el impacto en H (siempre y cuando se esté en el contexto de un coloreo válido en G). De esta manera, comienza a generar un nuevo coloreo que parte de aquél obtenido por el algoritmo goloso. Luego de terminar de generarlo lo devuelve. \\

\indent El algoritmo de búsqueda local que implementamos parte de una solución obtenidad por el algoritmo contructivo goloso aleatorio que desarrollamos y detallamos en la sección anterior de este Trabajo Práctico. Se recibe un parámetro que será el valor $porcentaje$, utilizado a la hora de aplicar el algoritmo goloso como solución inicial.\\
\indent Una vez obtenida esta solución base, se procede a iterar explorando el espacio de las soluciones vecinas.\\
\indent Para ello, fue menester definir la vecindad entre soluciones. Definimos la vecindad de una solución en el contexto del algoritmo de búsqueda local como aquellos coloreos válidos de G que se obtienen de modificar el color de algún nodo de dicha solución por el color de alguno de los vecinos de ese nodo en el grafo H. Resultó natural definir la vecindad de esta manera pensando en el significado de Impacto de un coloreo de G en H, dado que lo que buscamos es que para un coloreo válido de G la mayor cantidad de nodos vecinos entre sí en H estén coloreados del mismo color.\\

\indent Una vez hechos estos comentarios, nos abocamos a describir el funcionamiento de nuestra implementación del algoritmo de búsqueda local para resolver el problema del coloreo de máximo impacto de G en H.\\
\indent Como mencionamos anteriormente, partimos de una solución obtenida mediante nuestra implementación del algoritmo constructivo goloso, ejecutado con el parámetro $porcentaje$ que está búsqueda local recibe por parámetro. Una vez obtenida esa solución, se comienza a recorrer sus vecinas.\\
\indent Para ello, se comienza a iterar en los nodos de H. Por cada nodo, se recorren sus vecinos en H y se analiza si se mejora el impacto cambiándole el color a ese nodo por el de alguno de sus vecinos (siempre y cuando dicho coloreo sea válido en G). En caso de que se obtuviese un mejor impacto, se actualiza la solución con el cambio propuesto y se continúa iterando en los vecinos de ese nodo, esta vez comparando la solución parcial con la nueva solución obtenida. Una vez recorridos todos los vecinos de ese nodo en H, se pasa a otro nodo y se repite el proceso, actualizando la solución en caso de ser necesario con los criterios mencionados.\\
\indent El resultado es entonces una solución que se obtuvo de ir modificando la primera solución obtenida con el algoritmo goloso. Por la manera en que fuimos operando, podemos estar seguros de que si la solución final es diferente a la solución inicial de la que se partió, entonces el impacto del tal coloreo de G en H es mayor al de la solución golosa. En el caso en que ninguna de las soluciones vecinas de la solución golosa mejore el impacto se devuelve entonces la solución inicial.\\
\indent Hay que tener en cuenta que al ser una búsqueda local, se termina \textit{cayendo} en un mínimo local, con lo que probablemente no se obtenga la solución óptima del problema.


\begin{algorithm}[H]
\caption{} 
\begin{codebox}
\Procname{$\proc{maximoImpactoLocal}(Grafo$ g$, Grafo$ h$, double $ porcentaje$)$}

\li
\li vector$<$unsigned int$>$ impactoGoloso  $\gets$ maximoImpactoGoloso(g,h,porcentaje)
\li unsigned int impactoParcial $\gets$ impactoGoloso[0];
\li vector$<$unsigned int$>$ coloreo(n);
\li
\li vector$<$unsigned int$>$ solucionFinal(n+1);
\li
\li \For i desde 1 hasta n \Do
    
\li 	coloreo[i] $\gets$ impactoGoloso[i]
    
    \End
\li

\li unsigned int nuevoImpacto $\gets$ 0
\li
\li	\For i desde 1 hasta n \Do
	
\li
\li		vector$<$unsigned int$>$ vecinos $\gets$ vecinos del nodo i en h
\li
\li 	\For j desde 1 hasta la cantidad de vecinos de i en h \Do

\li				unsigned int color = coloreo[vecinos[j]]
\li
\li				\If pintar al nodo i de $color$ es legal \Do			
\li						vector$<$unsigned int$>$ nuevoColoreo $\gets$ \quad $coloreo$
\li						nuevoColoreo[i]$ \gets$ \quad $color$
\li                		nuevoImpacto $ \gets$ \quad $h.impacto(nuevoColoreo)$
\li
\li                		\If nuevoImpacto $>$ impactoParcial \Do
                
\li                			coloreo[i] $\gets$ \quad $color$
\li                   			impactoParcial $\gets$ \quad $nuevoImpacto$
                   		\End
\li
                \End
        \End
    \End

\li
%\li		impactoGoloso[0] $\gets$ \quad $impactoParcial$
\li 		solucionFinal[0] $\gets$ \quad $impactoParcial$
\li
\li \For i desde 0 hasta n \Do
%\li		impactoGoloso[i+1]=coloreo[i]
\li		solucionFinal[i+1]=coloreo[i]
	\End    

\li
%\li return impactoGoloso
\li return solucionFinal
\End
\end{codebox}
\end{algorithm}

\subsection{Análisis de complejidad}

\indent Veamos la complejidad de maximoImpactoLocal. Primero se calcula una solución con maximoImpactoGoloso. Como mencionamos en el apartado correspondiente eso cuesta O(n*(n+m)+ $n^{3}$). \\
\indent Luego, se copia el coloreo que se obtuvo de maximoImpactoGoloso en O(n).\\
\indent A continuación se itera sobre la cantidad de nodos de H. En cada iteración se copian los vecinos del nodo en el que estamos ahora. Como dicho nodo puede tener a lo sumo n-1 vecinos, eso cuesta O(n). Luego,se itera sobre los vecinos del nodo y por cada vecino del nodo se decide en O(n+m) si se va a pintar el nodo del mismo color que su vecino. Dicha decisión se fundamenta en si cambiar el color genera un coloreo válido en G y si aumenta el impacto H. Entonces,el costo del ciclo interior cuesta O(n*(n+m)). Por lo tanto, el ciclo que lo engloba cuesta O($n^{2}$*(n+m)).\\
\indent Luego de terminar de iterar se guarda el coloreo parcial con un costo de O(n).\\
\indent Entonces, maximoImpactoLocal cuesta O(n*(n+m)+ $n^{3}$ +$ n^{2}$*(n+m)).\\

\subsection{Experimentación y Resultados}

\quad A continuación analizaremos el costo temporal de la búsqueda local. Al igual que antes, testeamos con grafos generados al azar, con grafos estrellas no uniformes y grafos web de 4 vértices.

\quad Se midieron los tiempos en corridas de  5 a 100 nodos con 50 repeticiones para cada cantidad de nodos y se las comparó con la cota teórica calculada anteriormente.


\begin{figure}[H]
	\centering
	\includegraphics[scale=0.6]{timingBLocal.png}
\caption{Costos}
\end{figure}

\quad

\quad Se observa que se cumple con la cota del peor caso calculada.

\quad A diferencia de la heurística golosa, se puede ver que con los tres tipos de grafos se obtuvieron diferentes resultados.

\quad Con la que menor costos temporales se obtuvo fue con los grafos generados al azar. En cambio, vemos que con los grafos estrella el costo aumenta considerablemente. En menor medida ocurre con los grafos de red de 4 nodos. 

\quad Esto se debe a que estos dos últimos tipos de grafos en general son menos \textit{densos} que los grafos generados aleatoriamente. Como nuestra heurística de búsqueda local busca cambiar el color de un nodo por otro color legal (de algún vecino en el grafo H) en estos grafos menos densos hay mayor opciones por lo que se itera y prueba más veces.

\quad Si bien las cotas teóricas de las heurísticas golosa y de búsqueda local son las mismas, las constante que las multiplica son distintas, siendo mayor la de esta última ya que requiere más cálculo computacional.
	\newpage	
	%grasp
	\section{Metaheurística de Grasp}

\subsection{Algoritmo}

\subsection{Análisis de complejidad}

\subsection{Experimentación y Resultados}

	\newpage
	%experimentacion
	\section{Experimentación General}


\subsection{Comparando con el Exacto}

\quad Vamos a comparar todas nuestras implementaciones y a calcular las distancias con respecto al algoritmo exacto. Debido al costo del exacto se realizaron tests de 5 nodos a 16 con 50 repeticiones para cada cantidad. Definimos distancia como la diferencia entre el resultado exacto y el de la heurística correspondiente. Esto nos da una idea de \textit{que tan mala es la heurística}.


\quad Comparamos primero con grafos generados aleatoriamente, luego con grafos estrellas no uniformes y grafos \textit{web}.


\begin{figure}[H]
	\centering
	\includegraphics[scale=0.6]{distancia-Azar.png}
\end{figure}

\begin{figure}[H]
	\centering
	\includegraphics[scale=0.6]{distancia-Star.png}
\end{figure}

\begin{figure}[H]
	\centering
	\includegraphics[scale=0.6]{distancia-Web.png}
\end{figure}

\quad Se ve claro como a medida de que se incrementan los nodos en los grafos la calidad de las soluciones obtenidas empeoran.

\quad Con respecto a la heuristica golosa es la peor de las heuristicas, sin embargo como se vera más adelante es la de menor costo temporal. El empeoramiento (distancia) es lineal a la solución exacta en las 3 familias de grafos.

\quad Con la búsqueda local pasa algo similar salvo que por los datos obtenidos podriamos decir que es prácticamente la mitad de peor que el goloso. 

\quad En cambio, con la metaheurística GRASP se obtiene distancias casi nulas con respecto a la solución exacta. Siendo en promedio menor a una para este rango de nodos de grafos que experimentamos (en las 3 familias).

\quad

\quad Ahora vamos a mostrar los datos anteriores pero comparando cada heuristica consigo misma fijándonos la distancia con la solución exacta.

\begin{figure}[H]
	\centering
	\includegraphics[scale=0.6]{distancia-Goloso.png}
\end{figure}

\quad Prácticamente el goloso se comporta igual con los tres tipos de grafos, es decir, que la calidad de las soluciones obtenidas no depende de estas familias de grafos. Podemos determinar que a pesar de la similitud, esta heurística es levemente peor con los grafos Web de 4 vértices.

\quad

\begin{figure}[H]
	\centering
	\includegraphics[scale=0.6]{distancia-BLocal.png}
\end{figure}

\quad Vemos un comportamiento menos estable que con el goloso con respecto a las distintas familias de grafos. Sin embargo la tendencia es clara, lineal. No se puede determinar una familia para el cual esta heurística es peor.

\begin{figure}[H]
	\centering
	\includegraphics[scale=0.6]{distancia-GRASP.png}
\end{figure}

\quad Esta metaheurística es la menos estable de las 3. Varia mucho la distancia, presentando varios picos. Creemos que se debe a los criterios de parada definidos aunque hay que tener en cuenta que el promedio máximo es 0.5 menor a 1. Se podria decir que esta metaheuristica se comporta peor con las familias de grafo Web de 4 vertices porque con los grafos \textit{más grandes} es la que produce mayor distancia de la solución exacta.

\quad



\subsection{Comparando con Grasp}

\quad Dejando de lado el algoritmo exacto se puede experimentar con una mayor cantidad de nodos en un tiempo razonable.

\quad Comparamos entre si las heuristicas golosa, búsqueda local y GRASP.

\quad 

\subsubsection{Distancia}

\quad Vamos a calcular la distancia con el mismo criterio de la sección anterior, pero ahora tomando como punto de referencia la solución de la metaheurística GRASP.

\begin{figure}[H]
	\centering
	\includegraphics[scale=0.6]{distancia-Grasp-Azar.png}
\end{figure}

\begin{figure}[H]
	\centering
	\includegraphics[scale=0.6]{distancia-Grasp-Star.png}
\end{figure}

\begin{figure}[H]
	\centering
	\includegraphics[scale=0.6]{distancia-Grasp-Web.png}
\end{figure}

\quad Se observa que en los grafos aleatorios la heurística constructiva golosa tiene un comportamiento lineal, en cambio en las otras familias es levemente exponencial.

\quad La heurística de búsqueda local tiene un comportamiento lineal con todas estas familias de grafos. Aunque en los grafos aleatorios es más uniforme, un poco menos estable en los grafos estrellas y mucho menos estable en los grafos Web de 4 vértices.

\quad Hay que tener en cuenta que usamos a GRASP como punto de referencia por lo que su distancia siempre es 0.

\quad Si nos fijamos en los valores obtenidos podemos ver como la calidad de las soluciones obtenidas empeoran pasando de una familia a otra. En este sentido las heuristicas golosa y de búsqueda local se comportaron igual. Donde mejor fue la calidad fue con la familia de grafos aleatorios. En un punto intermedio fue con los grafos estrella y donde peor calidad de solución se obtuvo fue en los grafos Web de 4 vértices.

\quad Estos resultados no se pudieron apreciar comparando con el método exacto debido a que no fue posible  poder experimentar con la misma cantidad de nodos ya que tomaria un tiempo no prudencial.

\subsubsection{Costo}

\quad Ahora analizaremos el costo temporal con las tres familias de grafos y comparando las heuristicas entre sí.

\begin{figure}[H]
	\centering
	\includegraphics[scale=0.6]{timingAzar.png}
\end{figure}

\begin{figure}[H]
	\centering
	\includegraphics[scale=0.6]{timingStar.png}
\end{figure}

\begin{figure}[H]
	\centering
	\includegraphics[scale=0.6]{timingWeb.png}
\end{figure}

\quad

\quad Debido a que GRASP arrojó resultados mucho mayores en comparación con las otras dos heurísticas, tuvimos que usar la escala logarítmica para poder compararlos mejor.

\quad No se pueden apreciar diferencias relevantes entre los tres gráficos.

\quad HABLAR DE RELACION CONSUMO DE TIEMPO VS CALIDAD DE LA SOLUCION. GRASP CONSUMO MUCHO CALIDAD MUY BUENA, GOLOSO NO CONSUME NADA CALIAD PESIMA, BUSQUEDA LOCAL MEJORA ALGO DEL GOLOSO PERO NO ALCANZA PORQUE SE QUEDA EN MINIMOS LOCALES. A PESAR DEL COSTO DE GRASP ES MUCHO MAS CONVENIENTE QUE EL METODO EXACTO.

	\newpage
	\bibliographystyle{plain}	
	\clearpage
	\addcontentsline{toc}{section}{Bibliography}
%	\bibliography{bib}	
\end{document}

