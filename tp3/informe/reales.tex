\section{Descripción de situaciones reales}
En este trabajo se presenta otra variante al problema de coloreo, conocida como Coloreo de M\'aximo Impacto (CMI).
Se define el \textbf{impacto} de un coloreo C sobre un grafo G = (V, E) como el n\'umero de aristas $vw$ $\in $ E tal que el color de $v$ es igual al de $w$.
Dados dos grafos G y H definidos sobre el mismo conjunto de v\'ertices, CMI consiste en encontrar un coloreo C de G que al ser aplicado en H maximice el impacto de C en H. 
En general, observamos que los problemas que se pueden modelar tiene las siguientes caracter\'isticas:
\begin{itemize}
	\item se tiene un mismo conjunto de elementos, con dos criterios para relacionarlos entre si;
	\item se quiere asignar (o particionar) los elementos de forma tal que se maximiza la cantidad de elementos relacionados seg\'un criterio, mientras que se respeta el otro.
\end{itemize}
	
Presentamos aqu\'i algunos ejemplos de esto aplicado a situaciones que pueden darse en la vida real.


\subsection{Ejemplo 1}
En una facultad, se tiene \textsl{n} materias que corresponden a alguna de \textsl{m} \'areas de estudio.
Las materias pueden tener horarios solapados, de manera que no siempre es posible utilizar el mismo aula para dar dos materias distintas el mismo d\'ia.
Adem\'as, cada materia necesita que en su aula se encuentren elementos de trabajo espec\'ificos, seg\'un el \'area a la que corresponda. 
A fin de poder equipar las aulas de la mejor manera reduciendo la cantidad de equipamiento necesario, se desea asignarlas a las materias de forma tal que se maximice la cantidad de materias del mismo area que se dicten en un mismo aula.

Podemos resolver el problema planteando dos grafos G y H con las materias como nodos, tales que:
\begin{itemize}
	\item en G, los nodos est\'an relacionados por una arista si las materias tiene horarios solapados;
	\item en H, los nodos est\'an relacionados si pertenecen al mismo \'area de estudio,  
	\item los colores asignados representan las aulas asignadas a las materias.
\end{itemize}
En este caso, un coloreo de m\'aximo impacto ser\'a aquel que en G no asigne las mismas aulas a materias con horarios solapados, mientras que en H, maximiza la cantidad de materias del mismo \'area que comparten el mismo aula.

\subsection{Ejemplo 2}
Un centro de salud tiene pacientes internados en distintas sedes.
Los enfermeros del centro deben atender a los pacientes en horarios espec\'ificos, por lo si dos pacientes coinciden en estos horarios, el mismo enfermero no puede tratarlos. Debido a esto, puede ser necesario que los enfermeros se muevan entre las distintas sedes.
A fin de minimizar este movimiento de personal, se desea asignar los enfermeros a los pacientes tal que se maximice la cantidad de pacientes que cada uno debe tratar en una misma sede.

Aqu\'i, en el modelado consideramos a los pacientes como nodos de los grafos G y H, tal que 
\begin{itemize}
	\item en G, los nodos est\'an relacionados si los pacientes necesitan ser atendidos en el mismo momento;
	\item en H, los nodos est\'an relacionados si los pacientes se encuentran en la misma sede.  
	\item los colores asignados representan a los distintos enfermeros
\end{itemize}
Entonces, un CMI representa la asignaci\'on de enfermeros a los pacientes de manera que ningun enfermero sea asignado a dos pacientes si estos requieren atenci\'on en el mismo momento, mientras que se maximiza la cantidad de pacientes que cada enfermero ve en cada sede. 
