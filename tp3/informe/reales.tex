\section{Descripción de situaciones reales}

\quad Para ciertos tipos de problemas que se suelen modelar con grafos, se usa coloreo de grafos. Estos problemas, en general, suelen ser donde para ciertas cosas chocan interesas o hay incompatibilidades. Modelandolo, un nodo que representa un objeto está relacionado a otro, es decir, son adyacentes, si pasa lo anteriormente dicho. Coloreando el grafo, los nodos vecinos tienen distinto color, donde el color representa algo oportuno.

\quad

\quad En el caso de este problema, se aplica cuando uno modela un problema con coloreo para el grafo G, cambian las relaciones y se obtiene un grafo H. Se quiere averiguar cómo impacta el cambio, buscando el máximo valor posible. Sirve, de cierta forma, para medir esto.