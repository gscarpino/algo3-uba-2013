\section{Metaheurística de Grasp}

\subsection{Algoritmo}

\begin{algorithm}[H]
\caption{} 
\begin{codebox}
\Procname{$\proc{maximoImpactoGrasp}(Grafo$ g$, Grafo$ h$, double $ porcentaje$, unsigned$ $int$ maxIteraciones$, unsigned$ $int$ maxIterSinMejora,$ unsigned$ $int$ maxRCL$)$}

\li vector$<$unsigned int$>$ res(n + 1)
\li res[0] \gets 0
\li
\li vector$<$unsigned int$>$ coloreo(n,1) // Todos los elementos valen 1
\li
\li \For i desde 0 hasta maxIteraciones \Do 
\li vector$<$vector$<$unsigned int$>>$ rcl(maxRCL)
\li 	\For k desde 0 hasta maxRCL \Do
\li 		rcl[k] \gets $maximoImpactoGoloso(g,h, porcentaje)$
		\End
\li
\li		unsigned int e \gets $índice de uno de los elementos de rcl elegido al azar$
\li
\li 	vector$<$unsigned int$>$ solBusqLocal \gets $maximoImpactoLocal(g,h,porcentaje,rcl[e])$
\li

\End
\end{codebox}
\end{algorithm}

\subsection{Análisis de complejidad}

\indent Analicemos la complejidad de maximoImpactoGrasp. Los primeros pasos del algoritmos son crear unos vectores igual a la cantidad de nodos de los grafos. Eso cuesta O(n) para cada creación de vector.\\
\indent Luego se itera maxIteraciones veces. El costo de cada iteración es el siguiente:\\
\indent Primero se calcular maxRCL veces soluciones con maximoImpactoGoloso, donde maxRCL la cantidad de restrictive candidates list, es decir la cantidad máxima de candidatos golosos a utilizar.Ese ciclo cuesta entonces O(maxRCL*(n*(n+m)+ $n^{3}$) de acuerdo a nuestro análisis de complejidad de maximoImpactoGoloso.\\
\indent A continuación, se elige pseudoaleatoriamente uno de esos candidatos.\\
\indent Una vez elegido un candidato, se aplica maximoImpactoLocal con dicha solución golosa.\\ Por lo que analizamos en la sección correspondiente, esto cuesta O(n*(n+m)+ $n^{3}$ +$ n^{2}$*(n+m)).\\
\indent Una vez hecho esto, se decide si se va a quedar con la nueva solución obtenida con maximoImpactoLocal y esto cuesta O(n).\\
\indent Luego, el ciclo cuesta :\\

O(maxIteraciones* [(maxRCL*(n*(n+m)+ $n^{3})$)+ (n*(n+m)+ $n^{3}$ +$ n^{2}$*(n+m))] )\\

 que además es la complejidad de maximoImpactoGrasp.\\


\subsection{Experimentación y Resultados}
